\documentclass[conference]{sig-alternate-05-2015}
\usepackage{color, xcolor, float, lscape, enumerate, graphicx, url, tabularx, multirow, xspace, hyperref}%times
\usepackage[font=bf, skip=0pt]{caption}
%\usepackage{titlesec}

\hypersetup{
  colorlinks,
  citecolor=blue,
  linkcolor=red,
  urlcolor=black}
\newcommand{\note}[1]{{\textcolor{blue}{[#1]}}}
\newcommand{\fixme}[1]{{\textcolor{red}{#1}}}
\newcommand{\citeme}{{\textcolor{red}{[?]}}\xspace}
\newcommand{\todo}[1]{{\textcolor{red}{[#1]}}}
\newcommand{\BfPara}[1]{{\noindent\bf#1.}\xspace}
\newcommand{\vi}{\vspace{5mm}}
\newcommand{\etal}{{\em et al.}\xspace}
\newcommand{\eg}{{\em e.g.,}\xspace}
\newcommand{\ie}{{\em i.e.,}\xspace}
\newcommand{\etc}{{\em etc}\xspace}



\usepackage{fancyvrb}
\usepackage{verbatim}
\begin{document}



\title{Proposal: How To Train Your Dragon}


\author{Adam Smith\\ as@ucf.edu \and Alan M. Turing  \\ at@ucf.edu \and Claude E. Shannon\\ ces@ucf.edu}

\maketitle


\section{Motivation \& Problem Statement}\label{sec:motivation}
This is an introduction part, in which you should include a clear motivation on the problem being addressed in this project (why should I care?). You will need to also define the problem in broad terms so that you can outline the motivation. 

As the motivation is made clear, you want to also state the problem more formally; in terms that an NLP expert will understand. 


\section{Related Work}\label{sec:related}
At this stage, you don't want to provide a comprehensive list of related work, but rather you want to consider this as a part in which you will provide a list of the resources that will be used to assist you conducting the project. Find out some of the related work that would be relevant to this project, and summarize how similar or different your work to them is. Even better, highlight in broad terms what would the $\delta$ you think you will achieve by this magnificent work be. 



\section{How to Train Your Dragon}\label{sec:design}
Here should be the actual proposal. What methods are you proposing? Describe the method you will use in designing your training method for your dragons. Relate to the problem statement. The description should be specific so that reproduction of the training method you used for your dragon are reproducible. Avoid copying others' work and others' style, and be creative. 

\section{Evaluation}\label{sec:evaluation}

In this section, you want to describe two things: the data that you will use for evaluating the method against the problem stated above, and the results. As for data, describe your source of dragons, in as much details as needed, but not too much that I won't have the time to read. Be realistic. 

Here, I know that you won't have results, so don't worry. Describe to me the evaluation metrics that you will use for evaluating the approach on the dataset above. Describe concisely the steps you propose to use for evaluation against those methods/metrics. The description should be intended for the non-expert so that she is able to reproduce the results of your work. 

A bonus would be if you could propose to compare your work against a baseline. 

\section{Expected outcomes and risk management}

Here is your chance to tell me what you expect of outcomes. 

Also you want to tell me what are the risks associated with the project, and how you plan to deal with them. 

\section{Plan and Roles of Collaborators}
Divide your project into components, and tell me who is going to work on what, how much time each will work on each item. Have a timeline for the project. Tasks may include coding, testing, evaluation and analysis, write-up, presentation, etc. 


\section{References}\label{sec:conclusion}
This space would be your chance to list the resources you used for the proposal. 

\bibliographystyle{ieeetr}
\bibliography{bib}


\end{document}
