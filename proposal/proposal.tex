\documentclass[conference]{sig-alternate-05-2015}
\usepackage{color, xcolor, float, lscape, enumerate, graphicx, url, tabularx, multirow, xspace, hyperref}%times
\usepackage[font=bf, skip=0pt]{caption}
%\usepackage{titlesec}

\hypersetup{
  colorlinks,
  citecolor=blue,
  linkcolor=red,
  urlcolor=black}
\newcommand{\note}[1]{{\textcolor{blue}{[#1]}}}
\newcommand{\fixme}[1]{{\textcolor{red}{#1}}}
\newcommand{\citeme}{{\textcolor{red}{[?]}}\xspace}
\newcommand{\todo}[1]{{\textcolor{red}{[#1]}}}
\newcommand{\BfPara}[1]{{\noindent\bf#1.}\xspace}
\newcommand{\vi}{\vspace{5mm}}
\newcommand{\etal}{{\em et al.}\xspace}
\newcommand{\eg}{{\em e.g.,}\xspace}
\newcommand{\ie}{{\em i.e.,}\xspace}
\newcommand{\etc}{{\em etc}\xspace}



\usepackage{fancyvrb}
\usepackage{verbatim}
\begin{document}



\title{Proposal: To Create a Chatbot for Detecting Hate Speech}


\author{Wesley Aldridge\\ waldridge@knights.ucf.edu \and Miles C. Crowe  \\ miles.crowe@knights.ucf.edu \and Mauricio De Abreu\\ mabreu@knights.ucf.edu}

\maketitle


\section{Motivation \& Problem Statement}\label{sec:motivation}
Hate speech is defined broadly as aggressive language targeted at a person for attributes typically beyond their control.  Traits such as nationality, gender, race, sexual preference or disability are popular targets of hate speech[*].  Aggression towards the victim may manifest as insults, personal attacks or even threats.  As online communication between people is becoming more commonplace, exposure to hate speech is almost certain to reach more people.  Unfortunate as it is, this allows the spread of underlying beliefs that contribute to such behavior.

This project aims to curtail the spread of such speech by detecting hateful dialog and providing an alert mechanism to raise events for handling the occurrence automatically as it happens.  Clearly stated, the chat bot would be a silent participant in a chat, regardless of the number of participants, which would monitor the text exchanges.  When hate speech is detected, the event would be passed to a host framework or external agent to handle the event.


https://www.dictionary.com/browse/hate-speech

Communication channels such as social media and com-
munication software channels such as discord, whatsapp,
slack, etc should be moderated in order to protect its users
against harassment and also to prevent hate speech to dis-
seminate. Human moderators are far from a ideal solution
from the cost perspective. The current advances on NLP
however allow building automated online hate speech de-
tection.
\color{red}
This is an introduction part, in which you should include a clear motivation on the problem being addressed in this project (why should I care?). You will need to also define the problem in broad terms so that you can outline the motivation.

As the motivation is made clear, you want to also state the problem more formally; in terms that an NLP expert will understand.
\color{black}

\section{Related Work}\label{sec:related}
\color{red}At this stage, you don't want to provide a comprehensive list of related work, but rather you want to consider this as a part in which you will provide a list of the resources that will be used to assist you conducting the project. Find out some of the related work that would be relevant to this project, and summarize how similar or different your work to them is. Even better, highlight in broad terms what would the $\delta$ you think you will achieve by this magnificent work be.
\color{green}


Challenge 1

https://homes.cs.washington.edu/~msap/pdfs/sap2019risk.pdf
\color{black}



\section{How to Train Your Dragon}\label{sec:design}
\color{red}Here should be the actual proposal. What methods are you proposing? Describe the method you will use in designing your training method for your dragons. Relate to the problem statement. The description should be specific so that reproduction of the training method you used for your dragon are reproducible. Avoid copying others' work and others' style, and be creative.
\color{black}

\section{Evaluation}\label{sec:evaluation}

\color{red}In this section, you want to describe two things: the data that you will use for evaluating the method against the problem stated above, and the results. As for data, describe your source of dragons, in as much details as needed, but not too much that I won't have the time to read. Be realistic.

Here, I know that you won't have results, so don't worry. Describe to me the evaluation metrics that you will use for evaluating the approach on the dataset above. Describe concisely the steps you propose to use for evaluation against those methods/metrics. The description should be intended for the non-expert so that she is able to reproduce the results of your work. 

A bonus would be if you could propose to compare your work against a baseline. 
\color{black}

\section{Expected outcomes and risk management}

\color{red}Here is your chance to tell me what you expect of outcomes.

Also you want to tell me what are the risks associated with the project, and how you plan to deal with them.
\color{black}

\section{Plan and Roles of Collaborators}
\color{red}Divide your project into components, and tell me who is going to work on what, how much time each will work on each item. Have a timeline for the project. Tasks may include coding, testing, evaluation and analysis, write-up, presentation, etc.
\color{black}

\section{References}\label{sec:conclusion}
\color{red}This space would be your chance to list the resources you used for the proposal.
\color{black}
\bibliographystyle{ieeetr}
\bibliography{bib}


\end{document}
